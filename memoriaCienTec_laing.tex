%%%%%%%%%%%%%%%%%%%%%%%%%%%%%%%%%%%%%%%%%%%%%%%%%%%%%%%%%%%%
\documentclass[a4paper,11pt,oneside]{article}
\usepackage[a4paper,vmargin={1.5cm,1.5cm},width=16cm]{geometry}
\usepackage[style=verbose-inote,doi=false,sortcites=true,block=space,backend=bibtex]{biblatex}
\usepackage[utf8]{inputenc}
\usepackage{textcomp}
\usepackage[english]{babel}
\usepackage{microtype}
\usepackage{lmodern}
\usepackage{graphicx}
\usepackage{fancyhdr}
\usepackage{booktabs}
\usepackage{eurosym}
\usepackage{mathptmx}
\usepackage[T1]{fontenc}
\usepackage{hyperref}
%% Added to help mimic structure.
\usepackage{tcolorbox}
\usepackage{soul}
\usepackage{color}
\usepackage{lastpage}

%%% HEADER
\setlength{\headheight}{1cm}
%\setlength{\headwidth}{20cm}
\setlength{\headsep}{0.5cm}
\pagestyle{fancyplain}
\fancyheadoffset[HR,HL]{2cm}
\fancyhf{}
\lhead{\raisebox{-0.4\height}{\includegraphics[height=0.9cm,keepaspectratio=true]{img/miniLogo}}}
\rhead{\fancyplain{}{\fontsize{10}{12} \selectfont \textbf{\underline{MEMORIA CIENT\'IFICO-T\'ECNICA DE PROYECTOS de I+D+I PARA J\'OVENES INVESTIGADORES}}}}
\cfoot{\thepage\, / parte A}
\renewcommand{\headrulewidth}{0pt} % remove lines
\renewcommand{\footrulewidth}{0pt}


%%%%%%%%%%%%%%%%%%%%%%%%%%%%%%%%%%%%%%%%%%%%%%%%%%%%%%%%%%%%
%% Hack to make math formulas bold in section titles
\makeatletter
\DeclareRobustCommand*{\bfseries}{%
  \not@math@alphabet\bfseries\mathbf
  \fontseries\bfdefault\selectfont
  \boldmath
}
\makeatother

%%%%%%%%%%%%%%%%%%%%%%%%%%%%%%%%%%%%%%%%%%%%%%%%%%%%%%%%%%%%
\def\thesection{\bf \textsf{\alph{section}}}

\bibliography{biblio}

\begin{document}
\input{Commands.tex}

\begin{tcolorbox}[colback=white,arc=0pt,outer arc=0pt,colframe=black,boxrule=0.6pt]
  \begin{center}
    Convocatoria 2014\\
    Proyectos de I+D+I para j\'ovenes investigadores sin vinculaci\'on o con vinculaci\'on temporal\\
    Direcci\'on General de Investigaci\'on Cient\'ifica y T\'ecnica\\
    Subdirecci\'on General de Proyectos de Investigaci\'on
  \end{center}
\end{tcolorbox}
\begin{tcolorbox}[colback=yellow,arc=0pt,outer arc=0pt,colframe=black,boxrule=0.6pt,left=0mm,right=0mm]
  \begin{center}
    AVISO IMPORTANTE\\
  \end{center}
    En virtud del art\'iculo 11 de la convocatoria \ul{\textbf{NO SE ACEPTAR\'AN NI SER\'AN SUBSABABLES MEMORIAS CIENT\'IFICO-T\'ECNICAS}} que no se presenten en este formato.\\
    \\
    \textbf{Lea detenidamente las instrucciones que figuran al final de este documento para rellenar correctamente la memoria cient\'ifico-t\'ecnica.}
    \\
  %\end{center}
\end{tcolorbox}
\vspace{3pt}
\begin{tcolorbox}[colback=yellow,arc=0pt,outer arc=0pt,colframe=black,boxrule=0.6pt,left=0mm]
  \textbf{Parte A: RESUMEN DE LA PROPUESTA/SUMMARY OF THE PROPOSAL}
\end{tcolorbox}

\vspace{12pt}

\noindent\textbf{INVESTIGADOR PRINCIPAL/Principal investigator} (Nombre y apellidos): Andrew Brian Laing\\
\vspace{12pt}

\noindent\textbf{INVESTIGADOR TUTOR/Tutor} (Nombre y apellidos): Juan Jos\'e G\'omez Cadenas\\
\vspace{24pt}

\noindent\textbf{T\'ITULO DEL PROYECTO:}\\
\textbf{ACR\'ONIMO:}\\
\textbf{RESUMEN} {\color{blue}{M\'aximo 3500 caracteres (incluyendo espacios en blanco):}}
\vspace{12pt}

\noindent\textbf{PALABRAS CLAVE:}\\
\vspace{12pt}

\noindent\textbf{TITLE OF THE PROJECT:}\\
\textbf{ACRONYM:}\\
\textbf{SUMMARY} {\color{blue}{Maximum 3500 characters (including spaces):}}
\vspace{12pt}

\noindent\textbf{KEY WORDS:}\\
\vspace{12pt}

\newpage
\setcounter{page}{1}
\cfoot{\fancyplain{}{\thepage\, / parte B}}

\begin{tcolorbox}[colback=yellow,arc=0pt,outer arc=0pt,colframe=black,boxrule=0.6pt,left=0mm]
  \textbf{Parte B: INFORMAC\'ON ESPEC\'IFICA DEL INVESTIGADOR TUTOR}
\end{tcolorbox}

\textbf{FINANCIACI\'ON P\'UBLICA Y PRIVADA (PROYECTOS Y/O CONTRATOS DE I+D+I) DEL INVESTIGADOR TUTOR} (repita la secuencia tantas veces como se precise hasta un máximo de 10 proyectos y/o contratos)

\begin{enumerate}
\item Researcher from the research team who participates in the grant: Juan Jos\'e G\'omez Cadenas.
  \begin{itemize}
  \item[] Grant reference: ERC. Advanced Grant 339787-NEXT 
  \item[] Title:  NEXT.
  \item[] Principal investigator: Juan Jos\'e G\'omez Cadenas
  \item[] Funding Agency: European Research Council (ERC)
  \item[] Duration: 01/02/2014 -- 01/02/2018
  \item[] Grant value: 2.8 M\euro
  \item[] Relation with this project: Same topic
  \item[] Project status: Granted
  \end{itemize}
\item Researcher from the research team who participates in the grant: Juan Jos\'e G\'omez Cadenas.
  \begin{itemize}
  \item[] Grant reference: CSD2008-00037 
  \item[] Title:  Canfranc Underground Physics (CUP)
  \item[] Principal investigator: Concepci\'on Gonz\'alez Garc\'ia 
  \item[] Funding Agency: Ministerio de Educaci\'on y Ciencia
  \item[] Duration: 01/09/2014 -- 31/12/2015
  \item[] Grant value: 5.0 M\euro
  \item[] Relation with this project: Same topic
  \item[] Project status: Granted
  \end{itemize}
\item Researcher from the research team who participates in the grant: Juan Jos\'e G\'omez Cadenas.
  \begin{itemize}
  \item[] Grant reference: FIS2012-37947-C04-01 
  \item[] Title:  Coordination of NEXT Project
  \item[] Principal investigator: Juan Jos\'e G\'omez Cadenas
  \item[] Funding Agency:  Ministerio de Econom\'ia y Competitividad
  \item[] Duration: 01/01/2013 -- 31/12/2014
  \item[] Grant value: 256000\euro
  \item[] Relation with this project: Same topic
  \item[] Project status: Granted
  \end{itemize}
\item Researcher from the research team who participates in the grant: Juan Jos\'e G\'omez Cadenas.
  \begin{itemize}
  \item[] Grant reference: FPA2009-13697-C04-04 
  \item[] Title:  Física Experimental de Neutrinos en el IFIC / Experimental neutrino physics at IFIC 
  \item[] Principal investigator: Juan Jos\'e G\'omez Cadenas
  \item[] Funding Agency:  Ministerio de Educaci\'on y Ciencia, 
  \item[] Duration: 01/01/2010 -31/12/2012.
  \item[] Grant value: 534437\euro. 
  \item[] Relation with this project: Very related
  \item[] Project status: Granted
  \end{itemize}
\end{enumerate}

\newpage
\setcounter{page}{1}
\cfoot{\fancyplain{}{\thepage\, de \pageref{LastPage} / parte C}}

\begin{tcolorbox}[colback=yellow,arc=0pt,outer arc=0pt,colframe=black,boxrule=0.6pt,left=0mm]
  \textbf{Parte C: DOCUMENTO CIENT\'IFICO}
\end{tcolorbox}

\noindent\textbf{C.1. PROPUESTA CIENT\'IFICA}
\subsection*{The state of the art}
\subsubsection*{Introduction}
Neutrinos, unlike the other fermions of the Standard Model of particle physics, could be Majorana particles, that is, indistinguishable from their antiparticles. The existence of Majorana neutrinos would have profound implications in particle physics and cosmology. 

If neutrinos are Majorana particles, there must exist a new scale of physics (at a level inversely proportional to the neutrino masses) that characterises underlying dynamics beyond the Standard Model. The existence of such a new scale provides the simplest explanation of why neutrino masses are so much lighter than the charged fermions. Indeed,  understanding the new physics that underlies neutrino masses is one of the most important open questions in particle physics and it could have profound implications in our comprehension of the mechanism of symmetry breaking, the origin of mass, and the flavour problem. 

The existence of Majorana neutrinos would imply that lepton number is not conserved, which could be the origin of the matter-antimatter asymmetry observed in the Universe. The new physics related to neutrino masses could provide a new mechanism to generate the asymmetry called leptogenesis. Although the predictions are model dependent, two essential ingredients must be confirmed experimentally: 1) the violation of lepton number and 2) CP violation in the lepton sector. 

The only practical way to establish experimentally that neutrinos are their own antiparticles and that lepton number is not conserved is the detection of neutrinoless double beta decay (\bbonu). This is a hypothetical, very slow nuclear transition in which a nucleus with $Z$ protons decays into a nucleus with $Z+2$ protons and the same mass number $A$, emitting two electrons that carry essentially all the energy released (\Qbb). The process can occur if and only if neutrinos are Majorana particles.

%%%%%%%%%%%%%%%%%%%%%%%%%%%%%%%%%%%%%%%%%%%%%%%%%%%%%%%%%%%%
\subsubsection*{The experimental landscape}
The detectors used in double beta decay searches are designed to measure the energy of the radiation emitted by a \bb\ source. In the case of \bbonu, the sum of the kinetic energies of the two released electrons is fixed by the mass difference between the parent and the daughter nuclei: $Q_{\bb} \equiv M(Z,A)-M(Z+2,A)$. However, due to the finite energy resolution of any detector, \bbonu\ events are reconstructed within an energy region centered around \Qbb, typically following a gaussian distribution (Region of Interest or ROI). Other processes occurring in the detector can fall in the ROI, becoming a background and compromising the expected sensitivity. It follows that \bbonu\ experiments require {\bf excellent energy resolution}, and good {\bf suppression of potential backgrounds}.

All double beta decay experiments have to deal with an intrinsic background, i.e. the standard process of a double $\beta$-decay with the emission of two neutrinos (\bbtnu), that can only be suppressed by means of good energy resolution. Cosmogenic backgrounds can be suppressed by underground opperation and those from natural radioisotopes present in the materials of the detector and its surroundings by careful screening and selection of materials. {\bf Additional experimental signatures} that allow the distinction between signal and background can further improve the S/N have been a major area of development for the next generation of experiments. Several other factors such as {\bf detection efficiency} or the {\bf scalability to large masses} must be also taken into account during the design of a double beta decay experiment.
 
\subsubsection*{Recent results}
The status of the field has been reviewed recently by the project tutor\footcite{INSS2014}. Three current-generation experiments, with fiducial masses in the range of the 100 kg, have recently published the \bbonu\ search results. These are: GERDA, a high resolution calorimeter using \GE\ diodes; KamLAND-Zen, a low resolution, high-mass, self-shielding liquid scintillator calorimeter, with \XE\ dissolved in the scintillator; and EXO-200, a liquid xenon (LXe) TPC. All the experiments published negative results and, therefore, an upper limit on the halflife of the \bbonu\ decay (\Tonu). This limit can be translated into a limit on the \emph{effective Majorana mass} of the electron neutrino defined as:
\begin{equation}
\mbb = \Big| \sum_{i} U^{2}_{ei} \ m_{i} \Big| \, ,
\end{equation}
%
where $m_{i}$ are the neutrino mass eigenstates and $U_{ei}$ are elements of the neutrino mixing matrix. \mbb\ is related to \Tonu\ through the equation:
\begin{equation}
(T^{0\nu}_{1/2})^{-1} = G^{0\nu} \ \big|M^{0\nu}\big|^{2} \ \mbb^{2} \, .
\label{eq:Tonu}
\end{equation}

Here, $G^{0\nu}$ is an exactly-calculable phase-space integral for the emission of two electrons and $M^{0\nu}$ is the nuclear matrix element (NME) of the transition, which has to be evaluated theoretically. The uncertainty in the NME affects the value of \mbb\ which can be obtained from \Tonu.
 
{\bf GERDA} \footcite{Agostini:2013mzu} has a resolution of $\sim$0.2 \% FWHM around the \Qbb\ of \GE. The specific background rate in the ROI is $10^{-2}$ \ckky\ and the total exposure deployed 21.6 kg $\times$ yr. The experiment sets a limit $\Tonu > 2 \times 10^{25}$~yr, which translates to a range for \mbb\ of $[258-649]$~meV. The lowest value of \mbb\ corresponds to the IBM2 NME set, while the highest value corresponds to the ISM set.

{\bf EXO} \footcite{Albert:2014awa} achieves an energy resolution of 3.6\% FWHM at \Qbb, and a background rate of $ 4 \times 10^{-3}\ckky$. The total exposure used for the published result is 100 kg$\cdot$~yr. They have published a limit on the half-life of \bbonu\ in \XE\ of $T_{1/2}^{0\nu}(\XE) > 2 \times 10^{25}$~yr (assuming background only). The limit translates into a range for \mbb\ of $[125-352]$~meV.

{\bf KamLAND-Zen} \footcite{TheKamLAND-Zen:2014lma} compensates for a worse energy resolution of 10\% FWHM at \Qbb\ with a very small background rate of $\sim 4 \times 10^{-4}$ \ckky. After an exposure of 108.8 kg$\cdot$~yr, they obtain a limit  $T_{1/2}^{0\nu}(\XE) > 2.6 \times 10^{25}$~yr, which translates into a range for \mbb\ of $[110-309]$~meV.

 \subsubsection*{Potential for discovery}
 %%%%%%
%\begin{figure}
%\centering
%\includegraphics[width=0.75\textwidth]{img/SensiCRR.png}
%\caption{The allowed \mbb\ region, as a function of the sum of the neutrino masses, assuming that 
%$\sum m_i = 0.32$~eV. The blue lines mark the sensitivity of EXO and KamLAND-ZEN, the xenon-based detectors currently leading the field. The red line shows the sensitivity of NEXT after 3 years operation, which gives the experiment a sizeable chance of making a discovery.} 
%\label{fig.mbb}
%\end{figure}
%%%%%%
 Several analyses from recent cosmological results suggest that the sum of the masses of the three neutrinos could be $\sim$0.3~eV\footcite{PhysRevLett.112.051303}. The PI and collaborators have demonstrated that, in this case, if the neutrino is a Majorana particle, then, $\mbb \sim [20-150]$~ meV \footcite{GomezCadenas:2013ue}. In this scenario, the sensitivity of GERDA is outside the ``cosmologically relevant region'' (CRR), while both EXO-200 and KamLAND-Zen have already explored a significant fraction of CRR {\em for the most optimistic NME set} (in the most pessimistic NME they have not yet explored any part of the CRR). 
 
% Clearly, the experimental effort to determine if the neutrino is a Majorana particle, far from being completed is, rather, in its infancy. To establish unambiguously that the neutrino is (or not) a Majorana particle, even in this favourable scenario in which the sum of the neutrino masses is relatively high, experiments must be sensitive to $\mbb \sim 20$~meV, {\em even for the most pessimistic NME} set. On the other hand, a xenon experiment probing a $\Tonu >2.6 \times 10^{25}$~yr, has chances of making a discovery.
% 
 
%%%%%%%%%%%%%%%%%%%%%%%%%%%%%%%%%%%%%%%%%%%%%%%%%%%%%%%%%%%%
\subsubsection*{The NEXT experiment and its innovative concepts}
\begin{figure}
\centering
\includegraphics[width=0.9\textwidth]{img/NEXT.png}
\caption{\small A drawing of the NEXT-100 detector showing its main parts.  The pressure vessel (PV),  (130 cm inner diameter, 222 cm length, 1cm thick walls, with a total mass of 1\,200 kg) is made of a radio pure steel-titanium alloy.
The inner copper shield (ICS), is made of ultra-pure copper bars, 12 cm thick, with a total mass of 9\,000 kg.The electrical system includes the field cage, cathode, EL grids and HV penetrators.
The light tube is made of thin teflon sheets coated with TPB (a wavelength shifter). 
The energy plane is made of 60 PMTs housed in copper enclosures (cans).
The tracking plane is made of MPPCs arranged into dice boards (DB). 
}
\label{fig.NEXT100}
\end{figure}

The \emph{Neutrino Experiment with a Xenon TPC} (NEXT)\footnote{\href{http://next.ific.uv.es/}{http://next.ific.uv.es/}} will search for the \bbonu\ of \XE\ high-pressure gas (enriched to 91\%) using a time projection chamber (\HPXE). The advantages of the technology are: 
a) {\bf excellent energy resolution}, with an intrinsic limit of about 0.3\% FWHM at \Qbb, close to that of \GE\ detectors; b)
{\bf tracking capabilities} that provide a powerful topological signature to discriminate between signal (two electron tracks with a common vertex) and background (mostly, single electrons); c)
{\bf a fully active and homogeneous detector}, with no dead regions; d) {\bf scalability} of the technique to larger masses; %e) the possibility of exciting the barium ion produced in the xenon decay from the fundamental state \TwoS\ to the state \TwoP, using a ``blue'' laser (493.54 nm), and observing the ``red light'' emitted in the transition from \TwoP to \TwoD, thus ``tagging'' the presence of a barium atom in the xenon gas, which cannot be produced by any known background. 

The design of NEXT-100 (Figure \ref{fig.NEXT100}) is optimised for energy resolution by using proportional electroluminescent (EL) amplification of the ionisation signal. The detection process involves using the prompt scintillation light from the gas as start-of-event time, drifting the ionisation charge to the anode by means of an electric field ($\sim0.3$~kV~cm$^{-1}$ at 15 bar) where secondary EL scintillation will be produced in the region defined by two highly transparent meshes, between which there is a field of $\sim20$~kV~cm$^{-1}$ at 15~bar. The detection of EL light provides an energy measurement (in the energy plane, made of photomultipliers (PMTs), located behind the cathode) as well as providing tracking through its detection a few mm away from production at the anode plane, via a dense array of silicon photomultipliers called the \emph{tracking plane}.

\subsubsection*{The NEW detector}
\label{sec.new}
%%%%%%%%%%
\begin{figure}
\centering
\includegraphics[height=9cm]{img/NEW.png}
\caption{The NEW apparatus.} \label{fig:NEW}
\end{figure}

The NEW (NEXT-WHITE) apparatus\footnote{The name honours the memory of the late Professor James White, one of the key scientists of the NEXT Collaboration.}, shown in Figure \ref{fig:NEW} is the first phase of the NEXT detector to operate underground. NEW 
%has a triple goal:
%
%\begin{enumerate}
%\item {\bf Technology}: it will validate the technological solutions adopted by NEXT-100.
%\item {\bf Radiopurity}: it will allow the NEXT collaboration an extra step in the implementation of a radiopure detector.
%\item {\bf Physics}: it will demonstrate with measurements of the \BI\ and \TL\ lines, as well as with the measurement of the \bbtnu\ spectrum, the physics capabilities of NEXT-100.
%\end{enumerate}
%
has a scale 1:2 in size (1:8 in mass) compared to NEXT-100. The energy plane contains 12 PMTs (20 \% of the 60 PMTs deployed by NEXT-100). The tracking plane technology consists of 30 Kapton Dice Boards (KDB) deploying 1800 SiPMs (also 20\% of the sensors). The field cage has a diameter of 50~cm and a length of 60~cm (the dimensions of the NEXT-100 field cage are roughly 1~m length and 1.2~m radius). 

NEW is a necessary step\footnote{As formally stated by the scientific committee of the LSC, who recommended its construction in 2013.} towards the construction of NEXT-100. It will validate the technological solutions adopted by the collaboration and, as discussed below, is essential in the definition of the project methodology. Furthermore, The NEXT background model is currently based on a sophisticated Monte Carlo simulation of all expected background sources in each part of the detector. NEW will allow the validation of the background model with data. 
%Last but not least, NEW operation will demonstrate with measurements of the \BI\ and \TL\ lines, as well as with the measurement of the \bbtnu\ spectrum, the physics capabilities of NEXT-100.

%Furthermore, the calibration of NEW with 
%sources of higher energy, will allow a precise study of the evolution of the resolution with the energy. 
%In particular it will be plausible to measure the resolution near \Qbb\ using a Thorium source, which provides 2.6 MeV gammas. Last, but not least, we intend to 
%reconstruct the spectrum of \bbtnu. Those events are topologically identical to signal events (\bbonu) and can be used to demonstrate with data the power of the topological signature. 
%
\subsubsection*{Discovery potential of NEXT-100}
The excellent resolution of NEXT (0.5 \% FWHM) and the combination of low radioactive budget and topological signature (which yields an expected background rate of $5 \times 10^{-4} \ckky$, will allow the NEXT-100 detector to reach a sensitivity to the \bbonu\ halflife of $\Tonu > 7 \times 10^{25}$~yr for a exposure of 300~kg$\times$yr. This translates to a range for \mbb\ of $[67-187]$~meV. Therefore NEXT-100 will have a substantial chance of making a discovery if the NME is sufficiently high.

\subsection*{Involvement of the PI in NEXT to date and its impact on the project}
\label{subSec:Past}
The PI of this project began his involvement in the experiment in October of 2012, first as a visiting researcher then as a contracted member of the IFIC group from March 2013. During his time as a memeber of the collaboration he has worked in various aspects of the experiment. 
\begin{figure}
  \centering
  \includegraphics[width=0.7\textwidth]{img/DemoSetup.jpg}
  \caption{\small The NEXT-DEMO prototype setup at IFIC.} \label{fig.DEMO}
\end{figure}
%%%%%%%%%% 
A $\sim$1.5~kg natural xenon prototype of next (NEXT-DEMO, shown in figure~\ref{fig.DEMO}) has been taking data for the last two years at IFIC. NEXT-DEMO is  equipped with an energy plane made of 19 Hamamatsu R7378A PMTs and a tracking plane made of 256 Hamamatsu SiPMs. The detector has been operating successfully for more than two years and has demonstrated: (a) very good operational stability, with no leaks and very few sparks; (b) good energy resolution; (c) track reconstruction with PMTs and with SiPMs coated with TPB; (d) excellent electron drift lifetime, of the order of 20 ms. Its construction, commissioning and operation has been instrumental in the development of the required knowledge to design and build the NEXT detector.

The data taken over the course of the last two years has resulted in the publication of papers on the basic function and effect of wavelength shifting\footcite{Alvarez:2012xda}, the detection of alpha particles\footcite{Alvarez:2012hu}, improved resolution and basic tracking using SiPMs\footcite{Alvarez:2013gxa} and the use of Xe X-rays for characterisation and energy resolution improvement\footcite{Lorca:2014sra}. The PI has been a major author and internal editor in all of these papers with further papers currently being written. Currently, the data driven prediction of the energy resolution of NEXT at \Qbb\ is $\sim$0.7\% (see figure \ref{fig.ERES}).
%%%%% 
\begin{figure}
  \centering
  \includegraphics[width=0.8\textwidth]{img/EResolution.png}
  \caption{\small Left: the full energy spectrum measured for electrons of 511 keV in the DEMO detector. Right the spectrum near the photoelectric peak for 662 keV electrons in NEXT-DBDM. The resolution at 662 keV is 1\% FWHM (0.5\% FWHM at \Qbb). The resolution extrapolated from 511 keV is 0.7\%.}\label{fig.ERES}. 
\end{figure}
%%%% 

\subsubsection*{Run coordination for NEXT-DEMO data taking}
The PI has served as the run coordinator of these data runs organising a weekly run coordination meeting with all members of the group to decide calibration, maintenance and data scheduling. Additionally, he was responsable for the definition of shift and data monitoring duties for the members of the group and the scheduling of shifts and training for the newer members.

\subsubsection*{Detector calibration}
Calibration of the PMTs and SiPMs of NEXT is an important first step
in the commissioning of the detector. Using NEXT-DEMO the PI has been
involved in the improvement and standardisation of the basic
methodology for the calibration of the PMTs, work which will be
extended to NEW and NEXT-100. As seen in figure\ref{fig:cal}-\emph{left}, the low light
PMT spectra are fitted to a function which models the pedestal and a
number of related Gaussians representing the probability of detection
of one or multiple photoelectrons. Moreover, the PI suggested a backup
method for the calibration of the SiPMs in case of high electronic
noise (improving the calibration of certain channels and, thus, the
tracking plane as a whole) using the PTC
method\footcite{Janesick:2001}. This method determines the conversion
gain of the sensor by determining the gradient of a plot relating
variance and mean signal of the detection region
in which photon shot noise dominates (see figure \ref{fig:cal}-\emph{right}).
\begin{figure}
  \begin{center}
    \includegraphics[width=0.45\textwidth]{img/pmtFitEx}
    \includegraphics[width=0.45\textwidth]{img/Silinfit.png}
  \end{center}
  \caption{Sensor conversion gain extraction. (left) PMT response
    function, (right) a PTC curve for an example channel.}
  \label{fig:cal}
\end{figure}

\subsubsection*{Analysis of NEXT-DEMO data}
NEXT-DEMO data using atmospheric radiation as well as alpha particles
and gammas from radioactive sources has been recorded over the course
of the data taking runs mentioned above. The PI has been involved at
all levels of data processing and analysis. He has made improvements
to the data preprocessing algorithms particularly in the area of the
selection of SiPMs with signal and the exclusion of noisey
channels. In higher level analysis he has improved and developed a
series of cuts to further exlude events containing high noise levels,
possible sparks and particles of different types than those expected
in the run in question. He has also been active in the development of
the track reconstruciton algorithms currently considered the baseline
method for the experiment and using these algorithms has improved the
method of energy reconstruction by generalising the correction factors
to work with them. %Plot of some result been involved in?

\subsubsection*{Coordination of energy plane group for LSC based detectors}
As an additional coordination job, the PI has acted as coordinator of
the construction of the energy plane of NEW (and, ultimately, will
perform the same function for NEXT-100). Specifically, he has worked
to find appropriate optical coupling for the PMTs to their protective
sapphire windows and helped in the testing of said gel under
vacuum. Figure \ref{fig:gelRGA} shows a the results from a RGA measurement of
a sample of NyoGel OCK-451 showing little difference to the 'no
sample' measurement. Figure \ref{fig:gelPer} shows the improvement in light
collection efficiency achieved using this product to optically couple
a sapphire window to a PMT with the same window material as those to be
used in NEW/NEXT-100. He has also been responsible for collating the
costs and outstanding issues related to the radiopurity, cleaning and
construction for all mechanical and
electronic components of the energy plane.
\begin{figure}
  \begin{center}
    \includegraphics[width=0.65\textwidth,height=0.55\textwidth]{img/Opt_Gel_1}
  \end{center}
  \caption{RGA mesurement of NyoGel OCK-451 compared to a blank RGA.}
  \label{fig:gelRGA}
\end{figure}
\begin{figure}
  \begin{center}
    \includegraphics[width=0.65\textwidth]{img/gelPlot}
  \end{center}
  \caption{PMT photocurrent for pulses of calibrated LED with
    time. Curves compare the current in vacuum with the PMT behind a
    sapphire window with and without an optical coupling gel (NyoGel~OCK-451).}
  \label{fig:gelPer}
\end{figure}

\subsection*{Proposed aims of the project}
\label{subSec:Prop}
The proposal will involve the extension and formalisation of the work
currently undertaken by the PI. He will continue to coordinate the
development and construction of the Energy plane of NEW and will
update and develop the run coordination protocols for application to
data taking at LSC. Moreover, he will
generalise his work on detector calibration to involve all aspects of
the calibration of the detector energy measurement, optimising
techniques used in DEMO and developing others to meet the challenges
of the larger scale detectors. Additionally, he will begin R\&D using
NEXT-DEMO for a potential upgrade using next-generation SiPMs as both tracking and
energy measurement sensors.

\subsubsection*{Coordination of NEW energy plane construction}
The main mechanical and electronic designes for the energy plane of
NEW are already decided and under construction. The PI will continue
as the coordinator ensuring the the efficient marriage of all sections
of the energy plane and their integration with the other sections of
the detector.

NEW and, particularly the energy plane group, has reached the
construction stage and the role of the coordinator has become even
more importante. he will be required to coordinate the arrival of the
pieves which have already been sent for construction, their testing
under pressure and vacuum conditions, their cleaning to eliminate
surface radioisotope contamination and the construction and
integration of all components (Main components shown in figure \ref{fig:EPlane}). Additionally, the optical gel must
undergo radiopurity testing invloving the definition of the
methodology to be used for its application to ensure that the
measurement best represents the material to be used. The PI will work
directly with the mechanical, electronic and radiopurity experts to
ensure the efficient completion of all tasks.
\begin{figure}
  \begin{center}
    \includegraphics[width=0.65\textwidth]{img/EP.jpg}
  \end{center}
  \caption{Major energy plane components of NEW and NEXT-100.}
  \label{fig:EPlane}
\end{figure}
Cleaning of the components will take place in LSC and Gran Sasso
laboratory in Italy. The PI will coordinate with the facilities to
organise the delivery of each item to the appropriate place for
cleaning and, ultimately, to LSC for integration with the other
detector components.

\subsubsection*{Run coordination}
The NEXT-DEMO run coordination, shift scheduling and data monitoring
protocol has served as a small scale test of the protocols which will
be implemented to ensure that the data taking of both NEW and NEXT-100
runs smoothly. more stuff

\subsubsection*{Calibration}
The calibration aims proposed are more general than those performed in
DEMO. The initial deliverable in NEW will be the definition of the LED
settings for the PMT calibration. NEW will have multiple available
LEDs and is, of course, much larger than DEMO. The studies will
optimise the calibration of the PMTs both using individual PMTs and
the array as a whole so that the response of the PMTs can be
understood and any position dependency in the calibration limited.

The more general calibration protocol of the detector will involve the
use of multiple radioactive sources. Determination of the variation of
response to X-ray deposits must be understood in general as a first
step. The PI will begin by optimising the methods published
in\footcite{Lorca:2014sra} using DEMO data and will then work with the rest of the
calibraiton group to study the required data and its best application
within the optimised code to determine the drift and detector
correction factors to correct the response and to calculate the energy
scale. This deliverable will also require interaction with simulation
groups so that the methodology is fully understood and its systematic
errors accurately estimated.

\subsubsection*{R\&D with DEMO}
In addition to the aims concerned with the development and
exploitation of NEW and NEXT-100 the PI will undertake an innovative
new project to test the posibility of replacing the the PMTs of the
energy plane with next-generation SiPMs with very low dark current and
noise.

SiPM technology has developed a great deal since the beginning of the
NEXt project. The current generation of sensors from SensL
technologies for example have dark currents as low as ?. This coupled
with improved electronics and the lower mass and higher radiopurity of
the materials used in their construction make SiPMs an attractive
technology for use in low background experiments like NEXT.
\begin{figure}
  \begin{center}
    \includegraphics[width=0.495\textwidth]{img/siliEng}
    \includegraphics[width=0.495\textwidth]{img/siliNEXT}
  \end{center}
  \caption{The proposed SiPM energy plane for R\&D with NEXT-DEMO.}
  \label{fig:siliNEXT}
\end{figure}
Part of the PI's time would be spent on the construction and
characterisation of the the new SiPMs and their integration into DEMO
and, ultimately, in the demonstration of the efficient detection of
the primary scintillation signal and of the possible energy resolution
achievable.
\begin{table}
  \begin{center}
    \begin{tabular}{c|c|c|c}
      \hline
      Objective & Short description & Dates & Expected costs\\
      \hline
      NEW energy plane construction & & Jan. 2015 -- August 2015 &
      loads\\
      Preparation of run coord. protocol & & Jan. 2015 -- August 2015
      & nuthin\\
      Definition of energy plane calibration & & August 2015 --
      October 2015 & loads\\
      Development of detector calibration & & Jan. 2015 -- November
      2015 & loads\\
      Construction of components for NEXT-100 & & 2016 & loads
    \end{tabular}
  \end{center}
  \caption{Main objectives, their time scale and cost.}
\end{table}

\noindent\textbf{C.2. IMPACTO ESPERADO DE LOS RESULTADOS}\\


\noindent\textbf{C.3. IMPLICACIONES \'ETICAS Y/O DE BIOSEGURIDAD}\\
The proposal involves no ethical nor biosecurity implications.

\end{document}